\section{Aufgabe 1}
Bearbeiten Sie die folgenden Aufgaben und protokollieren Sie Ihr Vorgehen mit-
hilfe der Vorlage.
Implementieren Sie ein C-Programm, das folgende Anforderungen erfüllt:
\begin{itemize}
	\item Eine Datei wird zum Lesen geöffnet; anschließend wird zuerst die zweite
	Hälfte und dann die erste H ̈alfte des Dateiinhalts auf dem Bildschirm ausgegeben.
	\item  Danach wird der Inhalt der Datei in eine neue Datei kopiert, wobei der Dateiname der
Quell- und der Zieldatei dem Programm als Argument übergeben	werden kann.
\item  Die letzten 10 Zeichen der ursprünglichen Datei werden ab der 11. Stelle der neuen Datei
kopiert. Das Dateiende der neuen Datei soll jetzt nach den verschobenen Daten sein (also
nach dem 21. Zeichen).
\item Der Inhalt der Datei soll auf dem Bildschirm ausgegeben werden.


\end{itemize}

	\subsection{Vorbereitung}
	Für testzwecke wir erstellen eine Datei mit text (gefühlt mit generiertem text)
	\textit{filetocopy}
	\subsection{Durchführung}
	\begin{itemize}
		\item Damit wir Dateinamen als argumente an unsere programm übergeben können, definieren wir in main
		Function die Parameter \command{main(int argc, char *argv[])}. Der Parameter
		\command{argc} beinhaltet anzahl der übergebenen Argumente + 1 (programm name) und
		\command{argv} beinhaltet alle übergebenen Argumente. Ab index 1 können wir auf die Argumenten Liste
		zugreifen.
		
\end{itemize}


	\subsection{Fazit}

\newpage
