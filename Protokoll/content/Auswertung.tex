\section{Aufgabe 1}
Bearbeiten Sie die folgenden Aufgaben und protokollieren Sie Ihr Vorgehen mit-
hilfe der Vorlage.
Implementieren Sie ein C-Programm, das folgende Anforderungen erfüllt:
\begin{itemize}
	\item Eine Datei wird zum Lesen geöffnet; anschließend wird zuerst die zweite
	Hälfte und dann die erste Hälfte des Dateiinhalts auf dem Bildschirm ausgegeben.
	\item  Danach wird der Inhalt der Datei in eine neue Datei kopiert, wobei der Dateiname der
Quell- und der Zieldatei dem Programm als Argument übergeben	werden kann.
\item  Die letzten 10 Zeichen der ursprünglichen Datei werden ab der 11. Stelle der neuen Datei
kopiert. Das Dateiende der neuen Datei soll jetzt nach den verschobenen Daten sein (also
nach dem 21. Zeichen).
\item Der Inhalt der Datei soll auf dem Bildschirm ausgegeben werden.


\end{itemize}

	\subsection{Vorbereitung}
	Für Testzwecke erstellen wir eine Datei mit Text (gefüllt mit generiertem text)
	\textit{filetocopy} und \textit{testtarget}
	\subsection{Durchführung}
	\begin{itemize}
		\item Damit wir Dateinamen als Argumente an unser Programm übergeben können, definieren wir in der
		main Function die Parameter \command{main(int argc, char *argv[])}. Der Parameter
		\command{argc} beinhaltet die Anzahl der übergebenen Argumente + 1 (Programm Name) und
		\command{argv} beinhaltet alle übergebenen Argumente. Ab Index 1 können wir auf die Argument Liste
		zugreifen. Als erstes Argument erwarten wir den Pfad zur Quelldatei und als zweites Argument erwarten wir
	den Pfad zur Zieldatei. Die darauffolgenden Argumente werden ignoriert.
	\item wir prüfen ob genug Argumente übergeben wurden, wenn nicht, dann bricht das Programm ab.
	\item wenn alle benötigten Argumente vorhanden sind führt das Programm fort.
	\item wir definieren zwei Variablen, wo wir die Pfade für Quell- und Zieldatei speichern.
	\begin{lstlisting}
char * quellpath;
char * zielpath;
\end{lstlisting}


Zudem definieren wir Variablen die den Quell- respektive den ZielFiledescriptor halten.
Ein File Descriptor ist ein Handler um mittels In-und Output Methoden mit dieser
File umzugehen. In C ist eine Abbildung als Integer üblich. Dabei geben
negative Werte an, dass eine Fehlerbedingung vorliegt.
\begin{lstlisting}
int filedescriptorQuelle;
int filedescriptorZiel;
\end{lstlisting}

Diese werden in den Methoden \command{read}, \command{write},
\command{close} benutzt um die Datei mit der gearbeitet wird zu spezifizieren.

\item Für beide Dateien definieren wir jeweils eine Variable Modus.
Dies sorgt dafür dass die Quelle
im Lesemodus geöffnet wird.
Das Ziel soll sowohl im Lese- als auch im Schreibmodus geöffnet werden.

\begin{lstlisting}
mode_t modeQuelle = S_IRUSR;
mode_t modeZiel = S_IRUSR | S_IWUSR;
\end{lstlisting}

\item Für Beide Dateien definieren wir jeweils Variable, die Länge der Datei speichert
\begin{lstlisting}
long long quellsize;
long long targetsize;
\end{lstlisting}

\item Als nächsten Schritt überprüfen wir die Anzahl der übergebenen Argumente.
Wurden nicht genügend Parameter übergeben, also nicht Quelle und Ziel, so soll
das Programm abbrechen. Zuvor übermitteln wir mittels \command{perror} den Fehler
über den \command{stderr} Output Stream.

\begin{lstlisting}
if (argc <= 2){
	perror("Error: too few arguments");
	exit(EXIT_FAILURE);
}
\end{lstlisting}

\item Nun ermitteln wir den absoluten Pfad anhand der Eingaben des Users.
Konnte der Pfad nicht aufgelöst werden, also z.b. wenn die Datei nicht
gefunden werden konnte, so müssen wir das Programm abbrechen.
Dazu prüfen wir den Rückgabeparameter von \command{realpath}.
Ist dieser NULL, so existiert die Datei nicht und wir geben den Fehler
aus.
Hierbei ist anzumerken dass der Eingabeparameter eine beliebige Fehleranmerkung
ist. \command{perror} gibt nämlich den mit dem in \command{errno} hinterlegten
Fehlercode aus.
\begin{lstlisting}
if(realpath(argv[1], quellpath) == NULL){
	perror("Error argv[1]:");
	exit(EXIT_FAILURE);
}
\end{lstlisting}
\item Der Ziel Pfad wird auf gleiche Weise aufgelöst wie Quellpfad.
\begin{lstlisting}
if(realpath(argv[2], zielpath) == NULL){
	perror("Error argv[2]:");
	exit(EXIT_FAILURE);
}
\end{lstlisting}
\item Als nächstes ermitteln wir die Dateigrößen für die Quell- und Zieldatei.
Dies geschieht in dem wir eine in \command{<sys/stat.h>} definierte Struktur initialisieren
\command{quellstat} und hier mit Hilfe der Funktion \command{stat}
die Informationen über die Datei speichern. Anschließend
speichern wir in der Variable \command{quellsize} die in \command{quellstat} enthaltene Information über
die Dateigröße \command{st_size}. Auf gleiche Weise ermitteln wir auch die Größe des Zieldatei.
\begin{lstlisting}
struct stat quellstat;
stat(quellpath, &quellstat);
quellsize = (long long) quellstat.st_size;

struct stat zielstat;
stat(zielpath, &zielstat);
targetsize = (long long) zielstat.st_size;
\end{lstlisting}

\item Wir legen eine Variable \command{char bufQuelle[quellsize];} an um den Dateinhalt
aus der Quelle hier zu speichern und weiter zu verwenden.

 \item Anschließend können wir beide Dateien mit der function \command{open} öffnen.
 Die function liefert einen filedescriptor (um den Filestream zu verwalten),
 wenn die Datei geöffnet werden konnte,
 sonst liefert sie den Wert -1. Das nutzen wir in der if abfrage um Fehler abzufangen
 und abzugeben.
 Der Filestream wird im Read-Only Modus geöffnet, mit den oben beschriebenen
 Zugriffsrechten.
 \begin{lstlisting}[language=C]
if ((filedescriptorQuelle = open(quellpath,
 O_RDONLY, modeQuelle)) == -1) {
perror("Error by opening quellpath:");
exit(1);
}
 \end{lstlisting}

\item Das selbe wird für das Ziel getan. Allerdings öffnen wir hier die
Datei im Lese- und Schreibmodus bzw. erzeugen die Datei wenn sie noch
nicht existiert. Die Modi der Operationen sind bereits oben beschrieben worden.
\begin{lstlisting}
if ((filedescriptorZiel = open(zielpath, O_RDWR
| O_CREAT, modeZiel))
	== -1) {
perror("Error by opening zielpath:");
exit(1);
}
\end{lstlisting}

\item Wir definieren eine Funktion \command{backupFileContent},
die uns hilft die Dateiinhalte in bufQuelle zu speichern
\begin{itemize}
	\item Die Methode nimmt 3 Argumente an.
	Der Integer descriptor gibt den übergebenen File Descriptor an.
	Der char Pointer buffer wird mit den gelesenen Daten beschrieben.
	Der Integer size gibt die Länge an die aus dem File gelesen wird an.
	In unserem Fall ist dies immer die gesamte Länge des Files.
	Im Fehlerfall, also bei einem negativen Wert, wird ein Fehler ausgegeben.
	Nach dem Methodenaufruf wurde somit die übergebene Datei in ein
	char Array gelesen.
	\begin{lstlisting}
void backupFileContent(int descriptor, char *buffer, int size) {
size_t error = read(descriptor, buffer, size);
if (error == -1) {
perror("Inhalt konnte nicht gelesen werden bzw. Datei ist leer.");
}
}
	\end{lstlisting}

\end{itemize}

\item Nach dem Methodenaufruf wurde die bufQuelle mit dem Inhalt der
gelesenen Datei gefüllt.
\command{backupFileContent(filedescriptorQuelle, bufQuelle, quellsize);}

\item Nun erfolgt die Ausgabe der Datei. Dabei soll zunächst die zweite
Hälfte und dann die erste Hälfte ausgegeben werden.
Hierzu definieren wir eine Funktion die diese Ausgabe übernimmt.
\command{printHalfFile()}

\begin{itemize}
	\item Die Methode nimmt 2 Eingabeparameter.
	Der Integer filesize hält die Größe der auszugebenden
	Dateien.
	Der char Buffer hält den auszugebenden Text.

	Innerhalb der Methode wird zunächst die Hälfte der Größe ermittelt.
	Hierbei ist es möglich dass der Buffer einer ungerade Anzahl an
	Zeichen hat. In diesem Fall, damit wir kein Zeichen verlieren, müssen
	wir mittels \command{ceil} das Ergebnis aufrunden (aus <math.h>).
	Dieser Schritt wird nur für die zweite Hälfte getan.
	Die Hälfte, ohne runden, wird nun für die erste Hälfte ausgeführt.
	Mit diesen zwei Größen erstellen wir uns 2 char Arrays die den Text
	halten werden.

	Mittels \command{strncpy} speichern wir uns nun die zweite Hälfte aus
	der bufQuelle. Hierbei startet der Pointer, der Kopierschritt, an der mittleren
	Stelle (+ firsthalf) für die Länge von secondhalf (den Rest des Strings).
	Ziel ist bufpartone.
	Mittels \command{write} geben wir den Text nun aus. Die Konstante
	\command{OUT} repräsentiert den Wert 1, welcher den \command{stdout}
	File Stream darstellt. Auf diesen Stream möchten wir schreiben (dies sind
	die zu sehenden Ausgaben). bufpartone gibt den zu schreibenden String aus und
	secondhalf die Länge.
	Für den bufparttwo wird ähnliches getan, nur das beim Kopieren des Strings
	der Zeiger an der ersten Stelle anfängt und bis zur Mitte läuft.
	Auch dies wird dann ausgegeben.

	\begin{lstlisting}
void printHalfFile(int filesize , char *bufQuelle){
int secondhalf = ceil(filesize/2);
int firsthalf = filesize/2;
char bufpartone[secondhalf];
char bufparttwo[firsthalf];

strncpy(bufpartone, bufQuelle + firsthalf, secondhalf);
write(OUT,bufpartone,secondhalf);
strncpy(bufparttwo, bufQuelle, firsthalf);
write(OUT,bufparttwo, firsthalf);
}
\end{lstlisting}

\end{itemize}

\item Als nächstes definieren wir eine Funktion \command{copyFile}
um die 10 letzten Zeichen aus der Quelldatei zu lesen und in die Ziel Datei ab der 11. Stelle zu schreiben.
\begin{itemize}
		\item die Funktion nimmt als erstes Argument ein Filedescriptor für das Ziel Document,
		als zweiten Argument die Größe der ZielDatei, als dritten Argument die Größe des Quelldatei
		und als viertes Argument einen pointer auf ein chararray, das den Inhalt der Quelldatei beinhalten soll
		\item mit der Funktion \command{lseek} verschieben wir den cursor an die
		gewünschte Stelle.
		Dabei gibt die Zahl 10 die Stellen an um die wir den cursor verschieben
		möchten, 10 laut Aufgabe. Das \command{SEEK_SET} gibt an ab welcher Position wir
		den cursor verschieben möchten. \command{SEEK_SET} bezieht sich dabei auf
		den Dateibeginn. Wenn die Verschiebung nicht erfolgreich war, wird ein Wert -1 zurückgeliefert.
		Dies nutzen wir in der if-Abfrage und brechen die funktion ab wenn der cursor nicht verschoben werden konnte.
		\item Nachdem der cursor in die Zieldatei verschoben wurde legen wir ein chararray an,

		\command{char bufeleventoend[targetsize - 10];},

		dass alle Zeichen des Ziels bis auf die ersten 10 halten soll.

		\item Mittels \command{backupFileContent} schreiben wir uns die
		Zeichen 11-Ende in das Array bufeleventoend. Da der cursor an Stelle 11
		geschoben ist liest der Filedescriptor auch erst ab dieser Stelle.
		Die Länge können wir um 10 dekrementieren.
		Das Array hält nun alle Zeichen 10-Ende.

	\begin{lstlisting}
int copyFile(int zielDiscr,
int targetsize, int  quellsize,
char *bufQuelle) {

if (lseek(zielDiscr, 10, SEEK_SET) == -1){
perror("Couldnt set pointer ");
return -1;
}

char bufeleventoend[targetsize - 10];
backupFileContent(zielDiscr, bufeleventoend, targetsize - 10);
...
	\end{lstlisting}

	\item Zunächst legen wir ein Array für die letzten 10 Zeichen
	der Quelle an.
	Dieses befüllen wir mittels \command{strncpy}. Dazu fangen wir das
	Kopieren an der 11-letzten Stelle für 10 Zeichen an. Das Array hält nun
	die letzten 10 Zeichen.

	Nun können wir den cursor im Ziel wieder auf Stelle 10 setzen. Ist dieser
	Schritt geglückt so befüllen wir die Datei mit den restlichen Inhalt.

	Mittels \command{write} wird nun ab der Stelle des Cursors zunächst das
	Array lastchars in die Datei des Zieldescriptors geschrieben.
	Anschließend schreiben wir die Zeichen aus bufeleventoend, die den Rest
	der Zeichen der Zieldatei halten wieder in die Zieldatei ein. Dieser
	Schritt ist nötig da der erste \command{write} Vorgang die alten Zeichen
	überschrieben hat.

	Nun ist die Datei wie gewünscht zusammengestellt worden.

	\begin{lstlisting}
...
/* In Zieldatei schreiben */
ssize_t bytes_written;

char lastchars[11];

strncpy(lastchars, bufQuelle + quellsize - 11, 10);

if (lseek(zielDiscr, 10, SEEK_SET) == -1){
perror("Couldnt set pointer ");
return -1;
}

bytes_written = write(zielDiscr, lastchars, 10);
bytes_written = write(zielDiscr, bufeleventoend,
sizeof(bufeleventoend));
}
\end{lstlisting}

\end{itemize}

\item in \command{main} rufen wir die beschriebene Funktion \command{printHalfFile} auf.
\item in \command{main} rufen wir die beschriebene Funktion \command{copyFile}

\item Nachdem die Funktionen ihre Aufgaben erfüllt haben, müssen wir die geöffneten Dateien schließen.
Dabei nutzen wir die function \command{close}, der wir einen filedescriptor übergeben. Die function
liefert einen Wert = 0 zurück wenn die Datei erfolgreich geschlossen wurde. Sonst geben wir einen Fehler aus.
\begin{lstlisting}
if ((close(filedescriptorQuelle)) == 0
 && close(filedescriptorZiel) == 0) {
 printf("Dateien wurden geschlossen!\n");
} else {
 perror("Datei konnte nicht geschlossen werden!\n");
}
\end{lstlisting}



\end{itemize} %end first lvl itemize%



	\subsection{Fazit}

\newpage
