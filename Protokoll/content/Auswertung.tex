\section{Aufgabe 1}
Bearbeiten Sie die folgenden Aufgaben und protokollieren Sie Ihr Vorgehen mit-
hilfe der Vorlage.
Implementieren Sie ein C-Programm, das folgende Anforderungen erfüllt:
\begin{itemize}
	\item Eine Datei wird zum Lesen geöffnet; anschließend wird zuerst die zweite
	Hälfte und dann die erste Hälfte des Dateiinhalts auf dem Bildschirm ausgegeben.
	\item  Danach wird der Inhalt der Datei in eine neue Datei kopiert, wobei der Dateiname der
Quell- und der Zieldatei dem Programm als Argument übergeben	werden kann.
\item  Die letzten 10 Zeichen der ursprünglichen Datei werden ab der 11. Stelle der neuen Datei
kopiert. Das Dateiende der neuen Datei soll jetzt nach den verschobenen Daten sein (also
nach dem 21. Zeichen).
\item Der Inhalt der Datei soll auf dem Bildschirm ausgegeben werden.


\end{itemize}

	\subsection{Vorbereitung}
	Für testzwecke wir erstellen eine Datei mit text (gefühlt mit generiertem text)
	\textit{filetocopy}
	\subsection{Durchführung}
	\begin{itemize}
		\item Damit wir Dateinamen als argumente an unsere programm übergeben können, definieren wir in main
		Function die Parameter \command{main(int argc, char *argv[])}. Der Parameter
		\command{argc} beinhaltet anzahl der übergebenen Argumente + 1 (programm name) und
		\command{argv} beinhaltet alle übergebenen Argumente. Ab index 1 können wir auf die Argumenten Liste
		zugreifen. Als erste Argument erwarten wir Pfad zu Quelldatei und als zweites Argument erwarten wir
	Pfad zu Zieldatei. Die darauffolgende Argumente werden ignoriert
	\item wir prüfen ob genug Argumente übergeben wurde, wenn nicht, dann bricht das program ab.
	\item wenn alle benötigte Argumente vorhanden sind führt das program fort.
	\item wir definieren zwei Variablen, wo wir die Pfade für Quell- und Zieldatei speichern.
	\begin{lstlisting}
char * quellpath;
char * zielpath;
\end{lstlisting}
\item Für Beide Dateien definieren wir jeweils Variable \command{}, die filedescriptor speichern wird,
es handelt sich hierbei um ein ...  der wird in methoden \command{read}, \command{write},
\command{close} benutzt um die Datei mit der gearbeitet wird zu spezifizieren.
	\begin{lstlisting}
int filedescriptorQuelle;
int filedescriptorZiel;
\end{lstlisting}
\item Für Beide Dateien definieren wir jeweils Variable, die ...

\begin{lstlisting}
mode_t modeQuelle = S_IRUSR;
mode_t modeZiel = S_IRUSR | S_IWUSR;
\end{lstlisting}
\item Für Beide Dateien definieren wir jeweils Variable, die Länge der Datei speichert
\begin{lstlisting}
long long quellsize;
long long targetsize;
\end{lstlisting}


\end{itemize} %end first lvl itemize%



	\subsection{Fazit}

\newpage
